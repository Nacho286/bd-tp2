\section{Diseño lógico}

\par A continuación vamos a incluir los esquemas json que armamos en base a los diagramas de diseño y explicaremos cómo obtuvimos los datos que se incluyen en cada documento.

\subsection{Cliente}
\par Lo que hicimos fue embeber los datos de las atracciones visitadas (por medio de las entradas registradas) junto con la fecha de la entrada.
De esta manera podemos consultar las atracciones que visitó un cliente de forma simple y rápida.
\begin{verbatim}
{
  "type": "object",
  "properties": {
    "dni": "integer",
    "nombre": "string",
    "apellido": "string",
    "direccion": "string",
    "telefono": "string",
    "atraccionesVisitadas": {
      "type": "array",
      "items": {
        "type": "object",
        "properties": {
          "idEntrada": "integer",
          "idAtraccion": "integer",
          "nombre": "string",
          "fecha": {
            "type": "string",
            "format": "date-time"
          }
        }
      }
    }
  }
}
\end{verbatim}

\subsection{Evento}
\par Lo que hicimos fue embeber Evento y Entrada en Locacion, para tener en la misma estructura todos los eventos y sus visitas.
Así podremos consultar por un evento y obtener las visitas, con fecha, que tuvo.
\begin{verbatim}
{
  "type": "object",
  "properties": {
    "idLocacion": "integer",
    "nombre": "string",
    "cuitEmpresa": "integer",
    "fechaInicio": {
      "type": "string",
      "format": "date-time"
    },
    "fechaFin": {
      "type": "string",
      "format": "date-time"
    },
    "visitas": {
      "type": "array",
      "items": {
        "type": "object",
        "properties": {
          "idEntrada": "integer",
          "dni": "integer",
          "precio": "integer",
          "numeroDeTarjeta": "integer",
          "fecha": {
            "type": "string",
            "format": "date-time"
          }
        }
      }
    }
  }
}
\end{verbatim}

\subsection{Parque}
\par Lo que hicimos fue embeber Entrada en Atracción y luego Atracción en Parque. Para lograr tener las atracciones (y las entradas a esas atracciones) en los parques.
Luego embebimos Parque en Locación y Entrada en Locación. De esta forma tenemos, además de lo antes mencionado, las entradas a los parques incluidas.
\par Con esto podemos consultar diréctamente por un parque y obtener sus datos informativos, sus visitas (con fecha) y atracciones (con visitas a ellas).
\begin{verbatim}
{
  "type": "object",
  "properties": {
    "idLocacion": "integer",
    "precio": "integer",
    "atracciones": {
      "type": "array",
      "items": {
        "type": "object",
        "properties": {
          "idAtraccion": "integer",
          "nombre": "string",
          "precio": "integer",
          "minimoEdad": "integer",
          "minimoAltura": "integer",
          "visitas": {
            "type": "array",
            "items": {
              "type": "object",
              "properties": {
                "idEntrada": "integer",
                "dni": "integer",
                "precio": "integer",
                "numeroDeTarjeta": "integer",
                "fecha": {
                  "type": "string",
                  "format": "date-time"
                }
              }
            }
          },
          "cantidadVisitas": "integer"
        }
      }
    },
    "visitas": {
      "type": "array",
      "items": {
        "type": "object",
        "properties": {
          "idEntrada": "integer",
          "dni": "integer",
          "precio": "integer",
          "numeroDeTarjeta": "integer",
          "fecha": {
            "type": "string",
            "format": "date-time"
          }
        }
      }
    }
  }
}
\end{verbatim}

\subsection{Factura}
\par Lo que hicimos en este caso fue embeber Entrada en Factura. De esta manera consultamos una factura y tenemos el monto total y los ids de entradas para poder asociar luego.
\begin{verbatim}
{
  "type": "object",
  "properties": {
    "numeroDeFactura": "integer",
    "dni": "integer",
    "fecha": {
      "type": "string",
      "format": "date-time"
    },
    "monto": "integer",
    "entradas": {
      "type": "array",
      "items": {
        "type": "object",
        "properties": {
          "idEntrada": "integer",
          "precio": "integer"
        }
      }
    }
  }
}
\end{verbatim}
